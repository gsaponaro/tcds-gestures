%!TEX encoding = UTF-8 Unicode

\section{Conclusions and Future Work}
\label{sec:conclusions}

We presented a model that allows a robot to interpret and describe the actions of external agents by reusing the knowledge previously acquired in an ego-centric manner.
In a developmental setting, the robot first learns the link between words and object affordances by exploring its environment.
Then, it uses this information to learn to classify the gestures and actions of another agent.
Finally, by fusing the information from the two probabilistic models, we show in our experiments that the robot can reason over affordances and words when observing the other agent.
Although the complete model only estimates probabilities of single words given the evidence, we showed that feeding these probabilities into a pre-defined grammar produces human-interpretable sentences that correctly describe the situation.
We also highlighted some interesting language-related properties of the model, such as:
congruent/incongruent conjunctions,
choice of appropriate synonym words,
describing object features with general words.

Our demonstrations are based on a restricted scenario~(see Sec.~\ref{sec:experimental_settings}), i.e., one human and one robot manipulating simple objects on a shared table, a pre-defined number of motor actions and effects, and a vocabulary of approximately~$50$ words to describe the experiments verbally.
However, one of the main strengths of our study is that it spans different fields such as robot learning, language grounding, and object affordances.
We also work with real robotic data, as opposed to learning images-to-text mappings~(as in many works in computer vision) or using robot simulations~(as in many works in robotics).

As future work, it would be useful to investigate how the model can extract syntactic information from the observed data autonomously, thus relaxing the bag-of-words assumption in the current model.
The possibility of adding new words to the vocabulary would be helpful, too.
This involves both creating the auditory perception of new acoustic patterns~(e.g., \cite{falstrom:2017:glu, vanhainen2014:icassp, vanhainen:2012:interspeech}) and incorporating the new symbols into our \AffWords{} model.
In the gesture recognition part, we plan to overcome the limitation of enforcing the human to be fully visible~(we currently remove the tabletop to prevent occlusions and performance drops in the human hand~3D estimation, see Sec.~\ref{sec:experimental_settings:gesture_recognition}).
One avenue for this is to run the robust~2D human joint estimator OpenPose with the color cameras of the robot~\cite{cao:2017:openpose-cpvr}, transform the hand coordinates to~3D with existing robot vision software~\cite{roncone:2016:rss}, then feed the resulting gesture features to the recognizer.
