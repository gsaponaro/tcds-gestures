%!TEX encoding = UTF-8 Unicode

\section{Conclusions and Future Work}

Future work:
\begin{itemize}
  \item possibility to add new vocabulary words to the system
\end{itemize}

Things we need to discuss:
\begin{itemize}
\item the effect variables in the BN include hand velocity. How does this relate to the gesture features in the HMM? Can we learn the connection?
\item in order to achieve the ideal case in Figure~\ref{fig:model} we would need to learn the complex relationship between hand movements and affordance variables:
  \begin{itemize}
  \item how can this be achieved in the ego-centric learning? Mixture of HMMs and BN? Would it be possible to learn the dependency structure in this complex model?
  \item assuming it is possible to learn in the ego-centric way, how can this be extended to the other agent? Coordinates will definitely be different. In the current study we could directly map the two because we were using the categorical variable ``action''.
  \end{itemize}
\end{itemize}

BELOW IS THE GLU TEXT

Within the scope of cognitive robots that operate in unstructured environments, we have discussed a model that combines word affordance learning with body gesture recognition. We have proposed such an approach, based on the intuition that a robot can generalize its previously-acquired knowledge of the world~(objects, actions, effects, verbal descriptions) to the cases when it observes a human agent performing familiar actions in a shared \hr{} environment. We have shown promising preliminary results that indicate that a robot's ability to predict the future can benefit from incorporate the knowledge of a partner's action, facilitating scene interpretation and, as a result, teamwork.

In terms of future work, there are several avenues to explore. The main ones are (i)~the implementation of a fully probabilistic fusion between the affordance and the gesture components~(e.g., the soft decision discussed in Sec.~\ref{sec:combination}); (ii)~to run quantitative tests on larger corpora of \hr{} data; (iii)~to explicitly address the correspondence problem of actions between two agents operating on the same world objects~(e.g., a pulling action from the perspective of the human corresponds to a pushing action from the perspective of the robot, generating specular effects).
