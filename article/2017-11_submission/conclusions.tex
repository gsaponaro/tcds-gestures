%!TEX encoding = UTF-8 Unicode

\section{Conclusions and Future Work}
\label{sec:conclusions}

We have presented a model that combines robot ego-centric learning about language and object affordances with the observation of external agents' gestures.
We have illustrated a novel probabilistic method to fuse these two sources of information, and we have shown a number of experiments where the robotic model reasons over affordances and words when observing another agent.
Interestingly, these predictions of words allow us to create human-interpretable sentences~(from a pre-defined grammar) which highlight the emergence of language properties in the model, such as:
congruent/incongruent conjunctions,
choice of appropriate synonym words. % when referring to the same entity in two consecutive sentences.
We make our human action data and probabilistic reasoning code publicly available.

We believe that our study has merit because it spans different fields such as robot learning, language grounding, and object affordances, and it does that with real robotic data, as opposed to learning images-to-text mappings~(like many works in computer vision) or using robot simulations~(like many works in robotics).
However, we reiterate that our scenario and data are restricted to a particular scenario~(see Sec.~\ref{sec:assumptions}), i.e., one human and one robot manipulating simple objects on a shared table, a pre-defined number of motor actions and effects, and a vocabulary of approximately~$50$ words to describe the experiments verbally.

In terms of future work,

The model presented in this study was created following a number of assumptions, which can be relaxed in future work (TODO turn into text, discuss how much to say here):
\begin{itemize}
\item the gesture recognition part relies on depth sensor hardware~(e.g., Kinect) and human skeleton tracking software. In our experience, the hand tracking is not reliable in the presence of a tabletop~(i.e., partially occluded human) as in Fig.~\ref{fig:action_examples}, so we had to record the same gestures without the table. We plan to incorporate a robust color camera-based human joint estimator such as OpenPose~\cite{cao:2017:openpose-cpvr};
\item the model does not learn the grammar to generate verbal descriptions from word probabilities, but it is specified manually;
\item the model does not permit to add new words to its vocabulary (refer to TODO) \cite{falstrom:2017:glu, vanhainen2014:icassp, vanhainen:2012:interspeech}
\item except for the gesture/action recognizer, the rest of our model has no time evolution $\rightarrow$ try dynamic Bayesian networks?
\item multiple object or multiple interlocutors are not possible;
\item results were qualitative, rather than having a numerical evaluation.
\end{itemize}

% OLD PERSONAL NOTES
% - the effect variables in the BN include hand velocity. How does this relate to the gesture features in the HMM? Can we learn the connection?
% - learn the complex relationship between hand movements and affordance variables:
%   - how can this be achieved in the ego-centric learning? Mixture of HMMs and BN? Would it be possible to learn the dependency structure in this complex model?
%   - assuming it is possible to learn in the ego-centric way, how can this be extended to the other agent? Coordinates will definitely be different. In the current study we could directly map the two because we were using the categorical variable Action.
