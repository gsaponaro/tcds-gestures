%!TEX encoding = UTF-8 Unicode

\section{Conclusions and Future Work}
\label{sec:conclusions}

We presented a model that allows a robot to interpret and describe the actions of external agents by reusing the knowledge previously acquired in an ego-centric manner.
In a developmental setting, the robot first learns the link between words and object affordances by exploring its environment.
Then, it uses this information to learn to classify the gestures and actions of another agent.
Finally, by fusing the information from the two probabilistic models, we show in our experiments that the robot can reason over affordances and words when observing the other agent; this can also be leveraged to do early action recognition~(see Sec.~\ref{sec:results:anticipation_effects}).
Although the complete model only estimates probabilities of single words given the evidence, we showed that feeding these probabilities into a pre-defined grammar produces human-interpretable sentences that correctly describe the situation.
We also highlighted some interesting language-related properties of the model, such as:
congruent/incongruent conjunctions,
choice of appropriate synonym words,
describing object features with general words.

Our demonstrations are based on a restricted scenario~(see Sec.~\ref{sec:experimental_settings}), i.e., one human and one robot manipulating simple objects on a shared table, a pre-defined number of motor actions and effects, and a vocabulary of approximately~$50$ words to describe the experiments verbally.
However, one of the main strengths of our study is that it spans different fields such as robot learning, language grounding, and object affordances.
We also work with real robotic data, as opposed to learning images-to-text mappings~(as in many works in computer vision) or using robot simulations~(as in many works in robotics).

In terms of \emph{scalability}, note that our \ac{BN} model can learn both the dependency scructure and the parameters of the model from observations.
The method that estimates the dependency structure, in particular, is sensitive to biases in the data.
Consequently, in order to avoid misconceptions, the robot needs to explore any possible situation that may occur.
For example, if the robot only observes blue spheres rolling, it might infer that it is the color that makes the object roll rather than its shape.
In order to scale the method to larger number of concepts, it would be necessary to scale the amount of data considerably, similarly to what is typically done in deep learning.
In models of developmental robotics, where this is neither practically feasible, nor desirable, we would need to devise methods that can generalize more efficiently from very few observations.

As future work, it would be useful to investigate how the model can extract syntactic information from the observed data autonomously, thus relaxing the bag-of-words assumption in the current model.
Another line of research would be to study how the model can guide the discovery of new acoustic patterns~(e.g., \cite{falstrom:2017:glu, vanhainen:2014:icassp, vanhainen:2012:interspeech}), and how to incorporate the newly discovered symbols into our \AffWords{} model.
This would release our current assumption of a pre-defined set of words.
